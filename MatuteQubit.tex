%%% Template created by Marcel Neunhoeffer & Sebastian Sternberg
%%%This is a more detailed protocol adapted from Aaron


% This defines the style 
\documentclass[a4paper,12pt]{article} 

% Sets the margins 
\usepackage[top = 2.5cm, bottom = 2.5cm, left = 2.5cm, right = 2.5cm]{geometry} 

% Unfortunately, LaTeX has a hard time interpreting German Umlaute. The following two lines and packages should help
\usepackage[T1]{fontenc}
\usepackage[utf8]{inputenc}

% The default setting of LaTeX is to indent new paragraphs
\usepackage{setspace}
\setlength{\parindent}{0in}

% The fancyhdr package let's us create nice headers
\usepackage{fancyhdr}

%This package helps with SI units
\usepackage{siunitx}

% With this command we can customize the header style
\pagestyle{fancy} 

% This makes sure we do not have other information in our header or footer
\fancyhf{} 

% \lhead puts text in the top left corner. \footnotesize sets our font to a smaller size
\lhead{\footnotesize Qubit Protocol}

% We want to put our page number in the center
\cfoot{\footnotesize \thepage} 



\begin{document}

% This command disables the header on the first page
\thispagestyle{empty} 

\begin{tabular}{p{15.5cm}} % This is a simple tabular environment to align your text nicely 
{\large \bf Qubit Protocol} \\
Matute Lab \\ Written By: David Peede  \\ Last Update: August 2020\\
\hline % \hline produces horizontal lines
\\
\end{tabular} % Our tabular environment ends here

\vspace*{0.3cm} % Now we want to add some vertical space in between the line and our title

\begin{tabular}{p{15.5cm}} % Everything within the center environment is centered
	{\Large Materials Needed:} 
	\vspace{2mm}
	\\ {Qubit Kit} \\ {Qubit Assay Tubes}\\ {Falcon Tube} \\ {Qubit Fluorometer} \\ {DNA or RNA Samples} 
		
\end{tabular}

\vspace{0.4cm}

\textbf{IMPORTANT NOTE}: The Qubit reagent is sensitive to light, so make sure that it is not exposed to light until you are ready to use it. 

\begin{enumerate}

\item To make the master mix pipette \SI{199}{\micro\liter} of Qubit buffer and \SI{1}{\micro\liter} of Qubit reagent per sample into a Falcon tube, vortex for 15 seconds, and then place into a dark environment (e.g. a closed drawer). Note: You need to make enough master mix for all your samples AND the two calibrations AND two standards, while also accounting for pipetting error. 
\begin{enumerate}
\item 6 DNA/RNA samples + 2 calibrations + 2 standards = 10 samples total
\item \SI{199}{\micro\liter} of Qubit buffer x 10 samples x 1.1 = \SI{2189}{\micro\liter} of total Qubit buffer
\item \SI{1}{\micro\liter} of Qubit reagent x 10 samples x 1.1 = \SI{11}{\micro\liter} of total Qubit reagent 
\end{enumerate}
\item Pipette \SI{1}{\micro\liter} of DNA/RNA per sample into a Qubit assay tube and then add \SI{199}{\micro\liter} of the master mix into each Qubit assay tube. 
\item Prepare the two calibration tubes and two standard tubes:
\begin{enumerate}
\item Calibration 1 = \SI{190}{\micro\liter} of the master mix + \SI{10}{\micro\liter} of Standard 1
\item Calibration 2 = \SI{190}{\micro\liter} of the master mix + \SI{10}{\micro\liter} of Standard 2
\item Standard 3 = \SI{195}{\micro\liter} of the master mix + \SI{5}{\micro\liter} of Standard 2
\item Standard 2 = \SI{199}{\micro\liter} of the master mix + \SI{1}{\micro\liter} of Standard 2
\end{enumerate}
\item Vortex each Qubit assay tube for 15 seconds and then flick the tube to ensure that all the liquid is uniformly at the bottom of the Qubit assay tube. 
\item Select the setting on the Qubit fluorometer that reflects the Qubit kit.
\item Run calibrations one and two on the Qubit fluorometer. 
\item Run standards three and four on the Qubit fluorometer correcting for the volume of standard two in each solution. Note: The readings for standards three and four should be \SI{10}{\micro\gram}/\SI{}{\milli\liter} +/- \SI{5}{\micro\gram}/\SI{}{\milli\liter}.
\item Run your samples on the Qubit fluorometer correcting for the volume of DNA/RNA (\SI{1}{\micro\liter}) in each sample.
\end{enumerate}

\end{document}
